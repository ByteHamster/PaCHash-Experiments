\documentclass{article}
\usepackage[a4paper, margin=2cm]{geometry}
\usepackage{xcolor}
\usepackage{xspace}
\usepackage{booktabs}
\usepackage{dsfont}
\usepackage{footmisc}
\usepackage{marvosym}
\usepackage{amsmath}
\usepackage{hyperref}
\usepackage[capitalise,noabbrev]{cleveref}
\usepackage{tabularx}
\usepackage{listings}
\usepackage{multirow}
\usepackage{pgfplots}
\pgfplotsset{compat=newest}

\usepgfplotslibrary{groupplots}
\pgfplotsset{every axis/.style={scale only axis}}

\pgfplotsset{
  major grid style={thin,dotted},
  minor grid style={thin,dotted},
  ymajorgrids,
  yminorgrids,
  every axis/.append style={
    line width=0.7pt,
    tick style={
      line cap=round,
      thin,
      major tick length=4pt,
      minor tick length=2pt,
    },
  },
  legend cell align=left,
  legend style={
    line width=0.7pt,
    /tikz/every even column/.append style={column sep=3mm,black},
    /tikz/every odd column/.append style={black},
  },
  % move title closer
  legend style={font=\small},
  title style={yshift=-2pt},
  % less space on left and right
  enlarge x limits=0.04,
  every tick label/.append style={font=\footnotesize},
  every axis label/.append style={font=\small},
  every axis y label/.append style={yshift=-1ex},
  /pgf/number format/1000 sep={},
  axis lines*=left,
  xlabel near ticks,
  ylabel near ticks,
  axis lines*=left,
  label style={font=\footnotesize},       
  tick label style={font=\footnotesize},
  plotMaximumLoadFactor/.style={
    width=50.0mm,
    height=40.0mm,
  },
}

\title{PaCHash plot}
\date{}
\begin{document}
\definecolor{colorPaCHash}{HTML}{377EB8}
\definecolor{colorSeparator}{HTML}{000000}
\definecolor{colorPthash}{HTML}{000000}
\definecolor{colorLevelDb}{HTML}{FF7F00}
\definecolor{colorSilt}{HTML}{4DAF4A}
\definecolor{colorRocksDb}{HTML}{984EA3}
\definecolor{colorRecSplit}{HTML}{A65628}
\definecolor{colorUnorderedMap}{HTML}{F781BF}
\definecolor{colorCuckoo}{HTML}{E41A1C}
\definecolor{colorChd}{HTML}{444444}

\pgfplotscreateplotcyclelist{mycolorlist}{%
  {colorPaCHash, mark=diamond},
  {colorLevelDb, mark=square},
  {colorSilt, mark=o},
  {colorSeparator, mark=triangle},
  {colorRocksDb, mark=pentagon}
}
\pgfdeclareplotmark{pacman}{%
  \pgfpathmoveto{\pgfpointorigin}%
  \pgfpathlineto{\pgfqpointpolar{40}{1.2\pgfplotmarksize}}%
  \pgfpatharc{40}{320}{1.2\pgfplotmarksize}%
  \pgfpathlineto{\pgfpointorigin}%
  \pgfusepath{fill}
}
\pgfdeclareplotmark{flippedTriangle}{%
  \pgfpathmoveto{\pgfqpointpolar{-90}{1.2\pgfplotmarksize}}%
  \pgfpathlineto{\pgfqpointpolar{30}{1.2\pgfplotmarksize}}%
  \pgfpathlineto{\pgfqpointpolar{150}{1.2\pgfplotmarksize}}%
  \pgfpathclose%
  \pgfusepath{stroke}
}

% IMPORT-DATA objectSizeBlocksFetched figure-2.txt

\begin{figure*}[t]
    \centering
        \begin{tikzpicture}
            \begin{axis}[
              title={},
              plotMaximumLoadFactor,
                xlabel={Average object size},
                ylabel={\begin{tabular}{c}\textbf{I/O Volume}\\average [B/Query]\end{tabular}},
                ytick distance=1024,
                xtick distance=128,
                scaled ticks=false,
                cycle list name=mycolorlist,
            ]
            %% MULTIPLOT(a|ptitle)
            %% WITH
            %%    valuesForA(a) AS (values (2),(4),(8),(16),(32)),
            %%    valuesForObjectSize(size) AS (values (64),(960))
            %% SELECT
            %%    printf("$a$=%d theory", a) as ptitle,
            %%    a as a,
            %%    size as x,
            %%    4096.0*(1.0 + 1.0*size/4096.0 + 1.0/a) as y
            %% FROM valuesForA
            %% CROSS JOIN valuesForObjectSize
            %% GROUP BY MULTIPLOT, x
            %% ORDER BY MULTIPLOT, x


            %% MULTIPLOT(a|ptitle) SELECT
            %%    printf("$a$=%d real", a) as ptitle,
            %%    objectSize as x,
            %%    blocks_fetched*4096 as y,
            %%    MULTIPLOT
            %% FROM objectSizeBlocksFetched
            %% GROUP BY MULTIPLOT, x
            %% ORDER BY MULTIPLOT, x

            \legend{};
            \end{axis}
        \end{tikzpicture}
        \hfill
        \begin{tikzpicture}
            \begin{axis}[
              title={},
              plotMaximumLoadFactor,
                xlabel={Average object size},
                ylabel={\begin{tabular}{c}\textbf{Query Time}\\direct I/O [$\mu$s/Query]\end{tabular}},
                scaled ticks=false,
                xtick distance=128,
                legend columns=1,
                legend to name=bytesFetchedLegend,
                cycle list name=mycolorlist,
            ]
            %% MULTIPLOT(a|ptitle) SELECT
            %%    printf("$a$=%d", a) as ptitle,
            %%    objectSize as x,
            %%    1000000.0/queriesPerSecond as y,
            %%    MULTIPLOT
            %% FROM objectSizeBlocksFetched
            %% GROUP BY MULTIPLOT, x
            %% ORDER BY MULTIPLOT, x

            \end{axis}
          \end{tikzpicture}
          \hfill
          \begin{tikzpicture}[baseline={(0,-3)}]
            \ref{bytesFetchedLegend}
          \end{tikzpicture}
    \caption{Dependence of I/O volume and query time on the average object size $s$. Sizes are normal distributed with variance $s/5$, rounded to the next positive integer. Dotted lines show theoretic I/O volumes, while marks show measurements. Note that the measurements closely match the analysis. Using other distributions and plotting over the returned objects' sizes gives equivalent results.}
    \label{fig:bytesFetched}
\end{figure*}

\end{document}

